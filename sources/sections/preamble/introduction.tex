\section{Introduction} \label{chap:introduction}

Protocol attacks often involve attackers deploying a malicious smart contract.
Given these smart contracts are deployed before assets are being stolen, detection of these malicious smart contracts is of utmost importance.
Several detection bots are deployed on the Forta Network today, which identify malicious smart contracts using static and dynamic detection approaches.

The dynamic detection approach \href{\urlbotsimulation}{Smart Contract Simulation bot} attempts to simulate the execution of a malicious smart contract and observing whether suspicious state changes (e.g. TVL drop) occurs during the execution of the smart contract.

Given that malicious smart contracts are not source code verified, the bot attempts to guess the ABI and invoke the contract using a variety of heuristics.
This often fails (e.g. when the parameter list is unknown or the parameter requires to be of a certain value).

Fuzzing is the technique that can be utilized to execute smart contracts and - using a variety of techniques, such as taint analysis - make informed guesses on how to execute a contract to exhibit its behavior.
Unfortunately, these fuzzing tools have primarily been developed by smart contract auditors and operate on source code.

A possible avenue to work around this mismatch of having the bytecode of malicious smart contracts and fuzzing tools that require source code are decompilers.
Decompilers can turn byte code into valid source code.
This bounty is about assessing whether decompiling malicious smart contracts could increase the likelihood of successful execution and therefore successful detection.
